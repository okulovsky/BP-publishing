\documentclass[24pt,pdf,hyperref={unicode},aspectratio=169]{beamer}
\usepackage[utf8]{inputenc}
\usepackage[russian]{babel}
\usepackage{tikz}

\usetikzlibrary{calc}
\usetikzlibrary{shapes}

\tikzstyle{dedge} = [draw,thick,->]
\tikzstyle{edge} = [draw,thick,-]
\tikzstyle{gedge} = [draw=green,thick,-]
\tikzstyle{redge} = [draw=red,thick,-]
\tikzstyle{ver} = [circle, draw=black]
\tikzstyle{verg} = [circle, draw=black, fill=gray]
\tikzstyle{verb} = [circle, draw=black, fill=black, text=white]

\deftranslation[to=russian]{Lemma}{Лемма}
\deftranslation[to=russian]{Theorem}{Теорема}


\begin{document}

\section{Очередь с приоритетами}

\begin{frame}
Очередь с приоритетами содержит ключи с числовыми значениями, и допускает следующие операции: 
\begin{itemize}
\item Добавление ключа со значением
\item Удаление ключа 
\item Изменение значения по данному ключу
\item Извлечение ключа с минимальным (максимальным) значением.
\end{itemize}
\end{frame}

\section{Улучшенный алгоритм Дейкстры}

\begin{frame}
Проектируем интерфейс
\end{frame}

\begin{frame}
\begin{huge}
$$
\Theta\left(|V|T_{min} + |E|T_{upd}\right)
$$
\end{huge}
\end{frame}

\section{Бинарное дерево поиска}

\begin{frame}
\begin{columns}
\column{0.5\textwidth}
\begin{center}
\begin{tikzpicture}

\end{tikzpicture}
\end{center}
\column{0.5\textwidth}
\end{columns}
\end{frame}

\section{Бинарная куча}

\begin{frame}
\begin{columns}
\column{0.5\textwidth}
\begin{center}
\begin{tikzpicture}[x=1cm,y=-1cm]
\node[ver] (n0) at (0,0) {2};
\node[ver] (n1) at (-2,1) {6};
\node[ver] (n2) at (2,1) {4};
\node[ver] (n3) at (-3,2) {8};
\node[ver] (n4) at (-1,2) {7};
\path[dedge] (n0) -- (n1);
\path[dedge] (n1) -- (n3);
\path[dedge] (n1) -- (n4);
\path[dedge] (n0) -- (n2);
\end{tikzpicture}
\end{center}
\column{0.5\textwidth}
\uncover<+->{
Определение бинарной кучи:
\begin{itemize}
\item
Получена из полного бинарного дерева некоторой высоты $k$, из последнего уровня которого удалено несколько элементов справа налево
\item Для любых $u,v\in V$, если $(u,v)\in E$, то $val(u)\le val(v)$
\end{itemize}
}

\uncover<+->{
Свойства бинарной кучи:
\begin{itemize}
\item 
Минимальный элемент всегда находится в корне
\item 
Бинарная куча может быть реализована в виде массива. Корень кучи имеет индекс 1, дети элемента с индексом $i$ располагаются в индексах $2i$ и $2i+1$. 
\end{itemize}
}
\end{columns}
\end{frame}

\begin{frame}
\begin{center}
\begin{tikzpicture}[x=1cm,y=-1cm]
\node[ver] (n0) at (0,0) {\only<1-4>{3}\only<5->{1}};
\node[ver] (n1) at (-2,1) {5};
\node[ver] (n2) at (2,1) {\only<1-3>{4}\only<4>{1}\only<5-6>{3}\only<7->{2}};
\node[ver] (n3) at (-3,2) {7};
\path[dedge] (n0) -- (n1);
\path[dedge] (n1) -- (n3);
\path[dedge] (n0) -- (n2);

\uncover<2->{
\node[ver] (n4) at (-1,2) {6};
\path[dedge] (n1)--(n4);	
}

\uncover<3->{
\node[ver] (n5) at (1,2) {\only<3>{1}\only<4->{4}};
\path[dedge] (n2)--(n5);	
}

\uncover<6->{
\node[ver] (n6) at (3,2)
{\only<6>{2}\only<7->{3}};
\path[dedge] (n2)--(n6);	
}

\uncover<8->{
\node[ver] (n7) at (-3.5,3) {10};	
\path[dedge] (n3) -- (n7);
}


\end{tikzpicture}
\end{center}	
\end{frame}

\begin{frame}
\begin{center}
\begin{tikzpicture}[x=1cm,y=-1cm]
\node[ver] (n1) at (-2,1) {\only<1-5>{6}\only<6>{8}\only<7->{7}};
\node[ver] (n2) at (2,1) {\only<1-3>{4}\only<4->{6}};
\node[ver] (n3) at (-3,2) {\only<1-6>{7}\only<7->{8}};
\path[dedge] (n1) -- (n3);

\only<1-4>
{
\node[ver] (n4) at (-1,2) {8};
\path[dedge] (n1) -- (n4);
}

\uncover<1-2>
{
\node[ver] (n5) at (1,2) {6};
\path[dedge] (n2) -- (n5);
}

\uncover<1,3->
{
\node[ver] (n0) at (0,0) {\only<1,2>{2}\only<3>{6}\only<4>{4}\only<5>{8}\only<6->{6}};
\path[dedge] (n0) -- (n1);
\path[dedge] (n0) -- (n2);
}

\end{tikzpicture}
\end{center}	
\end{frame}


\end{document}