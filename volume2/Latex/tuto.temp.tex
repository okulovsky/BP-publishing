\documentclass[24pt,pdf,hyperref={unicode},aspectratio=169]{beamer}
\usepackage[utf8]{inputenc}
\usepackage[russian]{babel}
\usepackage{tikz}

\usetikzlibrary{calc}
\usetikzlibrary{shapes}

\tikzstyle{dedge} = [draw,thick,->]
\tikzstyle{edge} = [draw,thick,-]
\tikzstyle{gedge} = [draw=green,thick,-]
\tikzstyle{redge} = [draw=red,thick,-]
\tikzstyle{ver} = [circle, draw=black]
\tikzstyle{verg} = [circle, draw=black, fill=gray]
\tikzstyle{verb} = [circle, draw=black, fill=black, text=white]

\deftranslation[to=russian]{Lemma}{Лемма}
\deftranslation[to=russian]{Theorem}{Теорема}


\begin{document}
\section{Удаление элементов из бинарного дерева поиска}
\begin{frame}
\begin{center}
\begin{tikzpicture}[x=1cm,y=-1cm]

\node[ver] (n3) at (0,0) {
\only<1-7>{3}\only<8-10>{6}\only<11>{7}};
\node[ver] (n1) at (-4,1) {1};
\node[ver] (n9) at (4,1) {9};
\node[ver] (n10) at (6,2) {10};
\node[ver] (n2) at (-2,2) {1};

\path[dedge] (n9) -- (n10);
\path[dedge] (n1) -- (n2);
\path[dedge] (n3) -- (n1);
\path[dedge] (n3) -- (n9);	


\uncover<1-4>{
\node[ver] (n5) at (2,2) {5};
\node[ver] (n7) at (3,3) {7};
\node[ver] (n6) at (2.5,4) {6};
\node[ver] (n8) at (3.5,4) {8};

\path[dedge] (n9) -- (n5);	
\path[dedge] (n5) -- (n7);	
\path[dedge] (n7) -- (n6);	
\path[dedge] (n7) -- (n8);
}

\uncover<1-2>{
\node[ver] (n4) at (1,3) {4};
\path[dedge] (n5)--(n4);	
}

\uncover<5-9>{
\node[ver] (n7) at (2,2) {7};
\node[ver] (n8) at (3,3) {8};

\path[dedge] (n9) -- (n7);	
\path[dedge] (n7) -- (n8);
}

\uncover<5-6>{
\node[ver] (n6) at (1,3) {6};
\path[dedge] (n7) -- (n6);	
}

\uncover<10-11>{
\node[ver] (n8) at (2,2) {8};
\path[dedge] (n9) -- (n8);		
}
\end{tikzpicture}		

\only<2-3>{Удаление листа (4)}
\only<4-5>{Удаление узла с одним поддеревом (5)}
\only<6-8>{Удаление узла с двумя поддеревьями (3): случай, когда самый левый из элементов правого поддерева -- лист}
\only<9-11>{Удаление узла с двумя поддеревьями (6): случай, когда самый левый из элементов правого поддерева имеет одно поддерево}
\end{center}
\end{frame}

\end{document}
\end{document}