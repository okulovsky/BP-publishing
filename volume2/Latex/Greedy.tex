\documentclass[24pt,pdf,hyperref={unicode},aspectratio=169]{beamer}
\usepackage[utf8]{inputenc}
\usepackage[russian]{babel}
\usepackage{tikz}

\usetikzlibrary{calc}
\usetikzlibrary{shapes}

\tikzstyle{dedge} = [draw,thick,->]
\tikzstyle{edge} = [draw,thick,-]
\tikzstyle{gedge} = [draw=green,thick,-]
\tikzstyle{redge} = [draw=red,thick,-]
\tikzstyle{ver} = [circle, draw=black]
\tikzstyle{verg} = [circle, draw=black, fill=gray]
\tikzstyle{verb} = [circle, draw=black, fill=black, text=white]

\deftranslation[to=russian]{Lemma}{Лемма}
\deftranslation[to=russian]{Theorem}{Теорема}


\newcommand{\seg}[4]{
\draw[draw=#4] (#1,#3) -- (#2,#3);
\draw[draw=#4] ($(#1,#3)-(0,0.1)$) -- ($(#1,#3)+(0,0.1)$);
\draw[draw=#4] ($(#2,#3)-(0,0.1)$) -- ($(#2,#3)+(0,0.1)$);
}
\newcommand{\segb}[3]{\seg{#1}{#2}{#3}{black}}

\begin{document}

\section{Комбинаторные задачи}


\begin{frame}\frametitle{Задача разбиения}

Найти такое подмножество $B$ множества $A\subset \mathbb{N}$, что 

$$
\sum_{x\in B} x = \sum_{x\in A\setminus B} x
$$

\end{frame}

\begin{frame}\frametitle{Задача коммивояжера}


Найти такой цикл $C$ во взвешенном графе $G$, что каждая вершина $G$ входит в $C$ ровно один раз, и суммарный вес $C$ минимален из возможных.
\end{frame}


\section{Задача о планировании времени}

\begin{frame}
\begin{center}
\begin{tikzpicture}
\segb{1}{4}{0}
\segb{5}{8}{0}
\segb{9}{12}{0}

\segb{3}{6}{-1}
\segb{7}{10}{-1}

\end{tikzpicture}
\end{center}
\end{frame}

\begin{frame}
\begin{center}
\begin{tikzpicture}
\segb{1}{11}{0}
\segb{2}{4}{-1}
\segb{5}{7}{-1}
\segb{8}{10}{-1}
\end{tikzpicture}
\end{center}
\end{frame}

\begin{frame}
\begin{center}
\begin{tikzpicture}
\segb{1}{5}{0}
\segb{6}{10}{0}
\segb{4}{7}{-1}

\end{tikzpicture}
\end{center}
\end{frame}

\begin{frame}
\begin{center}
\begin{tikzpicture}[x=0.7cm]
\segb{1}{4}{0}
\segb{5}{8}{0}
\segb{9}{12}{0}
\segb{13}{16}{0}

\segb{7}{10}{-1}

\segb{3}{6}{-1}
\segb{3}{6}{-2}
\segb{3}{6}{-3}

\segb{11}{14}{-1}
\segb{11}{14}{-2}
\segb{11}{14}{-3}

\end{tikzpicture}
\end{center}
\end{frame}

\begin{frame}
\uncover<+->{
Доказательство корректности:

Пусть $s(x_i)$ -- начало промежутка $x_i$, $e(x_i)$ -- окончание.

Пусть $y_1,\ldots,y_n$ -- решение, найденное жадным алгоритмом, и $z_1,\ldots,z_m$ -- оптимальное решение.}

\uncover<+->{\begin{lemma}
Для любого $k$, $e(y_k)\le e(z_k)$.
\end{lemma}}

\only<+>{\begin{proof}
База индукции: жадный алгоритм выбирает $y_1$ так, что $e(y_1)$ -- минимально.

Шаг индукции: поскольку $e(y_{k-1})\le e(z_{k-1})$, то $z_k$ является допустимым для продолжения $y_1,\ldots,y_{k-1}$. $y_k$ -- элемент с минимальным $e$ из всех допустимых, следовательно, $e(y_k)\le e(z_k)$.
\end{proof}}

\uncover<+->{\begin{lemma}
$n\ge m$.
\end{lemma}}
\uncover<+->{\begin{proof}
Пусть $m>n$. Поскольку $e(y_n)\le e(z_n)$, то $z_{n+1}$ допустим для $y_1,\ldots,y_n$. Но тогда жадный алгоритм включил бы его в эту последовательность.
\end{proof}
}
\end{frame}

\section{Алгоритм Краскала}

\begin{frame}
\begin{center}
\begin{tikzpicture}[x=2cm, y=-2cm]

\node[ver] (n0) at (0,0) {};
\node[ver] (n1) at (-1,1) {};
\node[ver] (n2) at (-1,2) {};
\node[ver] (n3) at (0,3) {};
\node[ver] (n4) at (1,2) {};
\node[ver] (n5) at (1,1) {};

\only<1>{\path[edge] (n0) -- node[above] {1} (n1);}
\only<2->{\path[gedge] (n0) -- node[above] {1} (n1);}

\only<1-2>{\path[edge] (n2) -- node[below] {2} (n3);}
\only<3->{\path[gedge] (n2) -- node[below] {2} (n3);}

\only<1-3>{\path[edge] (n5) -- node[above] {3} (n0);}
\only<4->{\path[gedge] (n5) -- node[above] {3} (n0);}

\only<1-4>{\path[edge] (n1) -- node[above] {4} (n5);}
\only<5->{\path[redge] (n1) -- node[above] {4} (n5);}

\only<1-5>{\path[edge] (n1) -- node[above] {5} (n4);}
\only<6->{\path[gedge] (n1) -- node[above] {5} (n4);}

\only<1-6>{\path[edge] (n4) -- node[right] {6} (n5);}
\only<7->{\path[redge] (n4) -- node[right] {6} (n5);}

\only<1-7>{\path[edge] (n1) -- node[left]  {7} (n2);}
\only<8->{\path[gedge] (n1) -- node[left]  {7} (n2);}

\only<1-8>{\path[edge] (n3) -- node[below] {8} (n4);}
\only<9->{\path[redge] (n3) -- node[below] {8} (n4);}

\only<1-9>{\path[edge] (n2) -- node[below] {9} (n4);}
\only<10->{\path[redge] (n2) -- node[below] {9} (n4);}

\end{tikzpicture}
\end{center}
\end{frame}

\section{Корректность алгоритма Краскала}

\begin{frame}
\uncover<+->{\begin{theorem}
На каждом шаге алгоритма Краскала, последовательность ребер $e_1,\ldots,e_k$ является подмножеством минимальное остовного дерева.
\end{theorem}}
\uncover<+->{\begin{proof}
\uncover<+->{База индукции. Очевидно для $k=0$ и пустой последовательности.}

\uncover<+->{Шаг индукции. Пусть $F=\{e_1,\ldots,e_{k-1}\}$. По предположению индукции, существует минимальное остовное дерево $T$, содержащее $F$. Если $T$ содержит $e_k$, то шаг индукции выполняется.}

\uncover<+->{Если нет, то $T+e_k$ содержит цикл $C$. Этот цикл содержит некое ребро $p$, такое что $p$ не входит в $F$ (в противном случае, $F+e_k$ содержит цикл, и алгоритм Краскала не мог бы выбрать $e_k$ как продолжение $F$).}

\uncover<+->{Тогда $T-p+e_k$ является деревом. Учтем, что $w(e_k)\le w(p)$, поскольку жадный алгоритм выбрал $e_k$, а не $p$. Следовательно, $w(T-p+e_k)\le w(T)$, но $T$ -- оптимально, следовально, $T-p+e_k$ также оптимально.}
\end{proof}}

\end{frame}

\section{Алгоритм Дейкстры}

\begin{frame}
\begin{center}
\begin{tikzpicture}[x=3cm, y=3cm]

\node[ver] (n0) at (0,0) {{\uncover<2->{0}}};
\node[ver] (n1) at (2,0) {{\uncover<5->{4}}};
\node[ver] (n2) at (1,1) {{\uncover<3->{1}}};
\node[ver] (n3) at (1,-1) {{\uncover<4->{2}}};

\path [edge] (n0) -- node[above] {6} (n1);
\path [edge] (n0) -- node[above] {1} (n2) -- node[above] {4} (n1);
\path [edge] (n0) -- node[below] {2} (n3) -- node[below] {2} (n1);

\end{tikzpicture}
\end{center}
\end{frame}

\begin{frame}
\begin{center}
\begin{tikzpicture}[x=3cm, y=3cm]

\node[ver] (n0) at (0,0) {{\uncover<2->{0}}};
\node[ver] (n1) at (2,0) {{\uncover<4->{4}}};
\node[ver] (n2) at (1,1) {{\uncover<3->{2}}};
\node[ver] (n3) at (1,-1) {{\uncover<5->{?}}};

\path [edge] (n0) -- node[above] {2} (n2) -- node[above] {2} (n1);
\path [edge] (n0) -- node[below] {5} (n3);
\path [edge] (n2) -- node[left] {-4} (n3);
\end{tikzpicture}
\end{center}
\end{frame}

\section{Корректность алгоритма Дейкстры}

\begin{frame}

\uncover<+->{Обозначим {\tt track.Keys} через $S$, ${\tt track}[v].{\tt Price}$ через $p(v)$, {\tt start} через $v_0$.}

\uncover<+->{\begin{theorem}
Пусть все веса в графе неотрицательны.  Тогда на каждом шаге алгоритма для всех $v$ из $S$, $p(v)$ является длиной кратчайшего пути из $v_0$ в $v$.
\end{theorem}}
\uncover<+->{\begin{proof}
\uncover<+->{Индукция по количеству вершин в $S$.}

\uncover<+->{База индукции: очевидно для $v_0$.}

\uncover<+->{Шаг индукции. Пусть $v$ -- вершина, которую алгоритм добавляет в $S$ по ребру $(u,v)$. Пусть $P_u$ -- путь, найденный алгоритмом из $v_0$ в $u$, $P_v$ -- аналогичный путь для $v$. По предположению индукции $P_u$ является кратчайшим путем из $v_0$ в $u$. По выбору ребра $(u,v)$, $P_v$ является кратчайшим путем из $v_0$ в $v$ из тех, что проходят только через вершины из $S$.}

\uncover<+->{Предположим, что $P_v$ не является кратчайшим. Следовательно, существует другой путь $P$ с меньшей длиной. Поскольку $P_v$ -- кратчайший из путей, которые состоят только из вершин $S$, в $P$ должна быть вершина не из $S$. Обозначим первую такую вершину как $y$, предшествующую ей -- как $x$.} 

\end{proof}}
\end{frame}

\begin{frame}
\begin{center}
\uncover<1->{\begin{tikzpicture}[x=2cm, y=1cm]

\node[ver, fill=gray] (n0) at (0,0) {$v_0$};
\node[cloud, fill=gray] (n1) at (1,0) {$\ldots$};
\node[ver, fill=gray] (nu) at (3,0) {$u$};
\node[ver, fill=gray] (nx) at (2,-1) {$x$};
\node[ver] (ny) at (3,-1) {$y$};
\node[draw=black, cloud] (n2) at (4,-1) {$\ldots$};
\node[ver] (nv) at (5,0) {$v$};

\path[edge] (n0) -- (n1) -- node[above] {$P_u$} (nu) -- node[above] {$P_v$} (nv); 
\path[edge] (n1) -- node[below] {$P_x$} (nx) -- (ny) -- (n2) -- node[below]{$P$} (nv); 

\end{tikzpicture}}
\end{center}

$$
\uncover<2->{\mathcal{L}(P_v)}
\uncover<3->{=\mathcal{L}(P_u)+w(u,v)}
\uncover<4->{\overset{1}{\le} \mathcal{L}(P_x)+w(x,y)}
\uncover<5->{\overset{2}{\le} \mathcal{L}(P)}
$$

\uncover<4->{1) т.к. для добавления выбрана $v$, а не $y$}

\uncover<5->{2) т.к. все веса неотрицательны}

\end{frame}

\section{Жадный алгоритм для задачи разбиения}

\begin{frame}

$$
A=\{2,5,1,6,10,4\}
$$

$$
\begin{array}{c p{1cm} c}
   && \uncover<7->{1} \\
   && \uncover<6->{2} \\
\uncover<5->{4} && \uncover<4->{5} \\
\uncover<2->{10} && \uncover<3->{6} \\
\end{array}
$$
\end{frame}

\begin{frame}

$$
A=\{3,3,2,2,2\}
$$

$$
\begin{array}{c p{1cm} c}
   && \uncover<6->{2} \\
\uncover<4->{2} && \uncover<5->{2} \\
\uncover<2->{3} && \uncover<3->{3} \\
\end{array}
$$
\end{frame}

\begin{frame}

\uncover<+->{$$
P(B)=\max \left(\sum_{x\in B} x, \sum_{x\in A\setminus B} x \right)
$$}

\uncover<+->{$$
OPT(A) = \min _{B\subset A} P(B)
$$}

\uncover<+->{Если $c$ -- цена решения, найденного жадным алгоритмом для разбиения множества $A$, то $c\le 4 OPT(A)/3$}

\uncover<+->{$$
d(B)=\left|\sum_{x\in B} - \sum_{x\in A\setminus B} x\right|
$$}

\uncover<+->{Если $B$ -- решение, найденное жадным алгоритмом, то $d(B)=O\left(|A|^{-1}\right)$}

\end{frame}

\begin{frame}
\begin{center}
\begin{tikzpicture}[x=5cm, y=5cm]
\node (n0) at (0,0) {a};
\node (n1) at (1,0) {b};
\node (n2) at (0,1) {c};
\node (n3) at (1,1) {d};

\path[edge] (n0) --  node[above left] {1} (0.5,0.5) -- (n3);
\path[edge] (n0) -- node[left] {2} (n2);
\path[edge] (n0) -- node[below] {2} (n1);

\path[edge] (n2) --  node[below left] {100} (0.5,0.5)-- (n1);
\path[edge] (n1) -- node[right] {2} (n3);  
\path[edge] (n2) -- node[above] {2} (n3);  
\end{tikzpicture}
\end{center}
\end{frame}

\end{document}