\documentclass[24pt,pdf,hyperref={unicode},aspectratio=169]{beamer}
\usepackage[utf8]{inputenc}
\usepackage[russian]{babel}
\usepackage{tikz}


\tikzstyle{dedge} = [draw,thick,->]
\tikzstyle{edge} = [draw,thick,-]
\tikzstyle{ver} = [circle, draw=black]
\tikzstyle{verg} = [circle, draw=black, fill=gray]
\tikzstyle{verb} = [circle, draw=black, fill=black, text=white]


\deftranslation[to=russian]{Lemma}{Лемма}

\begin{document}
\section{Алгоритм Тарьяна}
\begin{frame}

\begin{center}
\begin{tikzpicture}[x=2cm,y=-2cm]

\only<1>{\node[ver] (n0) at (0,0) {0};}
\only<2-14>{\node[verg] (n0) at (0,0) {0};}
\only<15->{\node[verb] (n0) at (0,0) {0};}

\only<1-11>{\node[ver] (n1) at (1,1) {1};}
\only<12>{\node[verg] (n1) at (1,1) {1};}
\only<13->{\node[verb] (n1) at (1,1) {1};}


\only<1-2>{\node[ver] (n2) at (-1,1.5) {2};}
\only<3-9>{\node[verg] (n2) at (-1,1.5) {2};}
\only<10->{\node[verb] (n2) at (-1,1.5) {2};}

\only<1-7>{\node[ver] (n3) at (1,2) {3};}
\only<7>{\node[verg] (n3) at (1,2) {3};}
\only<8->{\node[verb] (n3) at (1,2) {3};}


\only<1-3>{\node[ver] (n4) at (0,3) {4};}
\only<4>{\node[verg] (n4) at (0,3) {4};}
\only<5->{\node[verb] (n4) at (0,3) {4};}


\path[dedge] (n0) -- (n1);
\path[dedge] (n0) -- (n2);
\path[dedge] (n1) -- (n2);
\path[dedge] (n1) -- (n3);
\path[dedge] (n2) -- (n3);
\path[dedge] (n2) -- (n4);
\path[dedge] (n3) -- (n4);

\end{tikzpicture}

\only<6->{4}\only<9->{, 3}\only<11->{, 2}\only<14->{, 1}\only<16->{, 0}

\end{center}

\end{frame}

\begin{frame}

\begin{center}
\begin{tikzpicture}[x=2cm,y=-2cm]

\only<1>{\node[ver] (n0) at (0,0) {0};}
\only<2->{\node[verg] (n0) at (0,0) {0};}


\only<1-2>{\node[ver] (n1) at (0,1) {1};}
\only<3-7>{\node[verg] (n1) at (0,1) {1};}
\only<8->{\node[circle,fill=red,text=white] (n1) at (0,1) {1};}


\only<1-3>{\node[ver] (n2) at (1,2) {2};}
\only<4->{\node[verg] (n2) at (1,2) {2};}


\only<1-6>{\node[ver] (n3) at (-1,2) {3};}
\only<7->{\node[verg] (n3) at (-1,2) {3};}


\only<1-4>{\node[ver] (n4) at (0,3) {4};}
\only<5>{\node[verg] (n4) at (0,3) {4};}
\only<6->{\node[verb] (n4) at (0,3) {4};}

\path[dedge] (n0) -- (n1);
\path[dedge] (n1) -- (n2);
\path[dedge] (n2) -- (n3);
\path[dedge] (n3) -- (n1);
\path[dedge] (n2) -- (n4);
\path[dedge] (n3) -- (n4);

\end{tikzpicture}
\end{center}
\end{frame}

\begin{frame}

\uncover<+->{\begin{lemma}
Если в орграфе нет циклов, то в нем есть вершина с нулевой степенью исхода.
\end{lemma}}

\uncover<+->{\begin{lemma}
Если $u$ -- вершина с нулевой степенью исхода, а $v_1,\ldots,v_n$ -- топологическая сортировка $G\setminus\{u\}$, то $v_1,\ldots,v_n,u$ -- топологическая сортировка $G$.
\end{lemma}}

\uncover<+->{\begin{lemma}
Если в орграфе нет циклов, то алгоритм Тарьяна найдет топологическую сортировку.
\end{lemma}}

\end{frame}

\begin{frame}
\uncover<+->{\begin{lemma}
Если в орграфе есть циклы, то алгоритм Тарьяна выдаст ошибку.
\end{lemma}}

\uncover<+->{\begin{proof}
Пусть $v_1,\ldots,v_n$ -- цикл. Тогда, начав поиск в глубину из одной из вершин $v_i$, алгоритм отметит эту вершину серым цветом. Затем, пройдя по вершинам $v_{i+1},v_{i+2},\ldots, v_{i-1}$, он вернется в вершину $v_i$, которая все еще отмечена серым цветом, и выдаст ошибку. 
\end{proof}}

\end{frame}


\end{document}
\end{document}