\documentclass[24pt,pdf,hyperref={unicode},aspectratio=169]{beamer}
\usepackage[utf8]{inputenc}
\usepackage[russian]{babel}
\usepackage{tikz}

\usetikzlibrary{calc}
\usetikzlibrary{shapes}

\tikzstyle{dedge} = [draw,thick,->]
\tikzstyle{edge} = [draw,thick,-]
\tikzstyle{gedge} = [draw=green,thick,-]
\tikzstyle{redge} = [draw=red,thick,-]
\tikzstyle{ver} = [circle, draw=black]
\tikzstyle{verg} = [circle, draw=black, fill=gray]
\tikzstyle{verb} = [circle, draw=black, fill=black, text=white]

\deftranslation[to=russian]{Lemma}{Лемма}
\deftranslation[to=russian]{Theorem}{Теорема}


\begin{document}

\section{Динамическое программировании в задаче разбиения}

\begin{frame}

Дано $A\subset \mathbb{N}$. Найти $B\subset A$ такое, что 

$$
\sum_{b\in B} b = \sum_{a\in A\setminus B} a
$$.

$$
2,3,4,5 \rightarrow 2+5=3+4 \rightarrow 2-3-4+5=0
$$
	
\end{frame}

\begin{frame}
\begin{center}
\begin{tikzpicture}[x=2cm,y=-2cm]
\uncover<+->{
\node (n0) at (0,0) {0};
}

\uncover<+->{
\node (n1) at (0,1) {5};
\path[dedge] (n0) -- node[right] {+5} (n1);
\node (n11) at (-1,1) {-5};
\path[dedge] (n0) -- node[above left] {-5} (n11);
}

\uncover<+->{
\node (n2) at (0,2) {1};
\path[dedge] (n1) -- node[left] {-4} (n2);

\node (n21) at (1,2) {9};
\path[dedge] (n1) -- node[above right] {+4} (n21);
}

\uncover<+->{
\node (n3) at (0,3) {-2};
\path[dedge] (n2) -- node[right] {-3} (n3);

\node (n31) at (-1,3) {4};
\path[dedge] (n2) -- node[above left] {+3} (n31);
}

\uncover<+->{
\node (n4) at (0,4) {0};
\path[dedge] (n3) -- node[left] {+2} (n4);

\node (n41) at (1,4) {-4};
\path[dedge] (n3) -- node[above right] {-2} (n41);
}
\end{tikzpicture}
\end{center}
\end{frame}

\begin{frame}
\begin{center}
\begin{tikzpicture}[x=0.5cm,y=-1cm]
\uncover<+->{}
\uncover<+->{
\node (v20) at (2,0) {2};
}
\foreach \y/\x/\py/\px in {
1/-1/0/2,
1/5/0/2,
2/-5/1/-1,
2/3/1/-1,
2/1/1/5,
2/9/1/5,
3/-10/2/-5,
3/0/2/-5,
3/-2/2/3,
3/8/2/3,
3/-4/2/1,
3/6/2/1,
3/14/2/9,
3/4/2/9}
{
\uncover<+->{
	\node(v\x\y) at (\x,\y) {\x};
	\path[dedge] (v\px\py) -- (v\x\y);	
}
}

\end{tikzpicture}
\end{center}
\end{frame}

\section{Сложность алгоритма ДП для задачи разбиения}

\begin{frame}
\begin{huge}
\uncover<+->{$$
O \left( n K\right)
$$}
\uncover<+->{$$
K=\sum_{a\in A} a\le n M
$$}
\uncover<+->{$$
M=\max_{a\in A} a
$$}
\uncover<+->{$$
O\left( n^2 M\right)
$$}

\end{huge}

\end{frame}


\end{document}