\documentclass[24pt,pdf,hyperref={unicode},aspectratio=169]{beamer}
\usepackage[utf8]{inputenc}
\usepackage[russian]{babel}
\usepackage{tikz}

\usetikzlibrary{calc}
\usetikzlibrary{shapes}

\tikzstyle{dedge} = [draw,thick,->]
\tikzstyle{edge} = [draw,thick,-]
\tikzstyle{gedge} = [draw=green,thick,-]
\tikzstyle{redge} = [draw=red,thick,-]
\tikzstyle{ver} = [circle, draw=black]
\tikzstyle{verg} = [circle, draw=black, fill=gray]
\tikzstyle{verb} = [circle, draw=black, fill=black, text=white]

\deftranslation[to=russian]{Lemma}{Лемма}
\deftranslation[to=russian]{Theorem}{Теорема}


\begin{document}


\section{Бинарное дерево поиска}

\begin{frame}
\begin{columns}
\column{0.5\textwidth}
\begin{center}
\begin{tikzpicture}[x=1cm,y=-1cm]
\node[ver] (n0) at (0,0) {5};
\node[ver] (n1) at (-2,1) {3};
\node[ver] (n2) at (2,1) {7};
\node[ver] (n5) at (1,2) {6};
\node[ver] (n6) at (3,2) {7};

\path[dedge] (n0) -- (n1);
\path[dedge] (n0) -- (n2);
\path[dedge] (n2) -- (n5);
\path[dedge] (n2) -- (n6);
\end{tikzpicture}
\end{center}
\column{0.5\textwidth}
\uncover<+->{
Определение бинарного дерева:
\begin{itemize}
\item
Каждый узел имеет не более двух детей
\item 
Для каждого узла $u$ левого поддерева узла $v$ выполнено $val(u) < val(v)$
\item 
Для каждого узла $u$ правого поддерева узла $v$ выполнено $val(u) \ge val(v)$
\end{itemize}
}

\uncover<+->{
Свойства бинарного дерева:
\begin{itemize}
\item В удачном случае, его высота $h$ составляет $\Theta(\log n)$
\item Поиск элемента длится $\Theta(h)$
\end{itemize}		
}	
\end{columns}
\end{frame}

\begin{frame}
\begin{center}
\begin{tikzpicture}[x=1cm,y=-1cm]
\node[ver] (n0) at (0,0) {5};
\node[ver] (n1) at (-4,1) {3};
\node[ver] (n2) at (4,1) {7};
\node[ver] (n3) at (2,2) {6};
\node[ver] (n4) at (6,2) {7};

\path[dedge] (n0) -- (n1);
\path[dedge] (n0) -- (n2);
\path[dedge] (n2) -- (n3);
\path[dedge] (n2) -- (n4);

\uncover<+->{}
	
\uncover<+->{
\node[ver] (n5) at (-6,2) {1};
\path[dedge] (n1) -- (n5);	
}

\uncover<+->{
\node[ver] (n6) at (-5,3) {2};
\path[dedge] (n5) -- (n6);	
}

\uncover<+->{
\node[ver] (n7) at (-2,2) {4};
\path[dedge] (n1) -- (n7);	
}

\uncover<+->{
\node[ver] (n8) at (-4.5,4) {2};
\path[dedge] (n6) -- (n8);	
}

\uncover<+->{
\node[ver] (n9) at (-3,3) {3};
\path[dedge] (n7) -- (n9);	
}

\uncover<+->{
\node[ver] (n10) at (-2.5,4) {3};
\path[dedge] (n9) -- (n10);	
}

\end{tikzpicture}
\end{center}
\end{frame}

\begin{frame}
\begin{center}
\begin{tikzpicture}[x=0.5cm,y=-1cm]

\node[ver] (n5) at (0,0) {5};
\node[ver] (n3) at (-4,1) {3};
\node[ver] (n7) at (4,1) {7};
\node[ver] (n9) at (6,2) {9};

\path[dedge] (n5) -- (n3);
\path[dedge] (n5) -- (n7);
\path[dedge] (n7) -- (n9);

\uncover<1>{
\node[ver] (n6) at (2,2) {6};
\path[dedge] (n7) -- (n6);	
}
\uncover<2>{}

\end{tikzpicture}		
\end{center}
\end{frame}


\begin{frame}
\begin{center}
\begin{tikzpicture}[x=1cm,y=-1cm]

\node[ver] (n3) at (0,0) {
\only<1-7>{3}\only<8-10>{6}\only<11>{7}};
\node[ver] (n1) at (-4,1) {1};
\node[ver] (n9) at (4,1) {9};
\node[ver] (n10) at (6,2) {10};
\node[ver] (n2) at (-2,2) {1};

\path[dedge] (n9) -- (n10);
\path[dedge] (n1) -- (n2);
\path[dedge] (n3) -- (n1);
\path[dedge] (n3) -- (n9);	


\uncover<1-4>{
\node[ver] (n5) at (2,2) {5};
\node[ver] (n7) at (3,3) {7};
\node[ver] (n6) at (2.5,4) {6};
\node[ver] (n8) at (3.5,4) {8};

\path[dedge] (n9) -- (n5);	
\path[dedge] (n5) -- (n7);	
\path[dedge] (n7) -- (n6);	
\path[dedge] (n7) -- (n8);
}

\uncover<1-2>{
\node[ver] (n4) at (1,3) {4};
\path[dedge] (n5)--(n4);	
}

\uncover<5-9>{
\node[ver] (n7) at (2,2) {7};
\node[ver] (n8) at (3,3) {8};

\path[dedge] (n9) -- (n7);	
\path[dedge] (n7) -- (n8);
}

\uncover<5-6>{
\node[ver] (n6) at (1,3) {6};
\path[dedge] (n7) -- (n6);	
}

\uncover<10-11>{
\node[ver] (n8) at (2,2) {8};
\path[dedge] (n9) -- (n8);		
}
\end{tikzpicture}		

\only<2-3>{Удаление листа (4)}
\only<4-5>{Удаление узла с одним поддеревом (5)}
\only<6-8>{Удаление узла с двумя поддеревьями (3): случай, когда самый левый из элементов правого поддерева -- лист}
\only<9-11>{Удаление узла с двумя поддеревьями (6): случай, когда самый левый из элементов правого поддерева имеет одно поддерево}
\end{center}
\end{frame}

\end{document}