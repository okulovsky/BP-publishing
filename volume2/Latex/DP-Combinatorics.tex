\documentclass[24pt,pdf,hyperref={unicode},aspectratio=169]{beamer}
\usepackage[utf8]{inputenc}
\usepackage[russian]{babel}
\usepackage{tikz}

\usetikzlibrary{calc}
\usetikzlibrary{shapes}

\tikzstyle{dedge} = [draw,thick,->]
\tikzstyle{edge} = [draw,thick,-]
\tikzstyle{gedge} = [draw=green,thick,-]
\tikzstyle{redge} = [draw=red,thick,-]
\tikzstyle{ver} = [circle, draw=black]
\tikzstyle{verg} = [circle, draw=black, fill=gray]
\tikzstyle{verb} = [circle, draw=black, fill=black, text=white]

\deftranslation[to=russian]{Lemma}{Лемма}
\deftranslation[to=russian]{Theorem}{Теорема}


\begin{document}

\section{Динамическое программировании в комбинаторике}

\begin{frame}
Задача: найти количество последовательностей нулей и единиц, не содержащих более $k$ нулей подряд, длины $n$.
\end{frame}

\begin{frame}


\begin{center}
\begin{tikzpicture}[x=3cm,y=-1.7cm]
\uncover<+->{
\node (p4) at (0,0) {$P(4)$};
}

\uncover<+->{
\node (p3) at (-1,1) {$P(3)$};
\draw[dedge] (p4) -- node[above left] {...1} (p3);
\node (p2) at  (0,1) {$P(2)$};
\draw[dedge] (p4) -- node[left] {..10} (p2);
\node (p1) at  (1,1) {$P(1)=2$};
\draw[dedge] (p4) -- node[above right] {.100} (p1);
}

\uncover<+->{
\node (p32) at (-2,2) {$P(2)$};
\draw[dedge] (p3) -- node[above left] {..1} (p32);
\node (p31) at  (-1,2) {$P(1)$};
\draw[dedge] (p3) -- node[left] {.10} (p31);
\node (p30) at  (0,2) {$P(0)=1$};
\draw[dedge] (p3) -- node[above right] {100} (p30);
}

\uncover<+->{
\node (p321) at (-3,3) {$P(1)$};
\draw[dedge] (p32) -- node[above left] {.1} (p321);
\node (p320) at  (-1,3) {$P(0)$};
\draw[dedge] (p32) -- node[above right] {10} (p320);
}
\end{tikzpicture}
\end{center}

\uncover<+->{$$
P(0)=1,\ P(1)=2,\ P(n)=\sum_{\min(0,n-k-1)}^{n-1} P(i)
$$}
\end{frame}


\end{document}